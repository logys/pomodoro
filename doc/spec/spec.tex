\documentclass{article}
\title{Especificación del pomodoro}
\author{Luis Antonio Sánchez Monterde}
\begin{document}
\maketitle
\section{Objetivo del proyecto}
\begin{itemize}
	\item Desarrollar un dispositivo que permita controlar una sesión de
		estudio.
\end{itemize}
\section{Metas}
\begin{itemize}
	\item Se desea controlar el tiempo de estudio con un pomodoro.
\end{itemize}

\subsection{Características de tiempo de estudio}
\begin{itemize}
	\item Iniciar una sesión de estudio.
	\begin{itemize}
		\item Arrancar la sesión en pomodoro.
		\item Indicar la finalización con un sonido.
		\item Apagado eficiente.
	\end{itemize}

	\item Controlar el dispositivo con un solo botón.
	\begin{itemize}
		\item Pulsación para cambiar entre pausa y play.
		\item Indicador para observar play o la pausa.
		\item Apagado eficiente.
	\end{itemize}
\end{itemize}

\subsubsection{Descripción de iniciar sesión de estudio}
Característica: Arrancar la sesión.
Con el objetivo de iniciar una sesión de estudio.
Como un estudiante.
Yo quiero presionar el botón una vez.

Característica: Indicar la finalización de la sesión.
Con el objetivo de Conocer si la sesión ha finalizado.
Como un estudiante.
Yo quiero escuchar una alerta.

Característica: Apagado eficiente.
Con el objetivo de ser eficiente,
Como usuario,
Quiero que el dispositivo entre en modo ahorro al finalizar la sesión.

\subsubsection{Descripción de controlar con un solo botón}
Característica: Cambiar entre pausa y play.
Con el objetivo de pausar y reanudar la sesión, 
Como usuario
Quiero pulsar el botón e intercambiar el estado entre pausa y play.

Característica: Indicador inicio de pausa o play.
Con el objetivo de saber en que estado se en encuentra el dispositivo,
Como usuario
Quiero que un indicador visual(led) parpadee a diferente ritmo para cada
cambio de estado durante unos segundos.
\end{document}
